\documentclass[a4paper,11pt]{jarticle}
\usepackage{ics-thesis}
\usepackage{amssymb}
\usepackage{graphicx}
\pagestyle{bachelorthesis}    % 卒論・規定の.styファイルを使う場合
%
\title{サイボーグインセクトの群れを用いて倒壊した建物内に捕らわれている被災者を探索するためのフロッキング制御手法の提案}
\author{北浦 直}
\supervisor{若宮 直紀 教授}
\deadline{2019年2月12日}
%
\begin{document}
	\titlepage    % 規定の.styファイルを使う場合
	\abstract     % 規定の.styファイルを使う場合
	%%%%%%%%%%%%%%%%%%%%%
	% 内容梗概本文
	%%%%%%%%%%%%%%%%%%%%%
	\keyword
	%%%%%%%%%%%%%%%%%%%%%
	% キーワード
	%%%%%%%%%%%%%%%%%%%%%
	\tableofcontents    % 目次
	%
	%%%%%%%%%%%%%%%%%%%%%
	% 本文
	%%%%%%%%%%%%%%%%%%%%%
	%
	\section{はじめに}
	災害が起きた場合,倒壊した建物内に被災者が取り残されることが起こりうる.
	倒壊した建物内に捕らわれた被災者の探索において,現在は災害救助犬やスコープカメラなどの人間の能力を補助するような手法が広く使われている.
	しかし,倒壊した建物内は人間や災害救助犬が入れないような環境であることが多い.%ここで例とかあったほうがいいかも
	また,救助活動の中で2次災害が起きてしまう事例も少なくない.
	そこで,現在の探索手法では探索できないような狭い空間を探索可能で,探索中の2次災害の危険性を低くすることができるサイボーグインセクトを用いた被災者探索の研究がなされている.
	
	サイボーグインセクトを被災者探索に活用するための
	\section{サイボーグインセクト}
	\section{サイボーグインセクトのモデル}
	\section{制御モデル}
	\subsection{フロッキング}
	\subsection{実装}
	\subsection{パラメータ設定}
	\section{実験結果}
	\section{おわりに}
	\begin{thebibliography}{99}
		%%%%%%%%%%%%%%%%%%%%%
		% 参考文献リスト
		%%%%%%%%%%%%%%%%%%%%%
		\item 
	\end{thebibliography}
\end{document}